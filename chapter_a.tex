\environment[envs/env_setting.tex]
% \disablemode[proofread]

% \showboxes

\starttext

% 单元标题,包装的是\chapter
\danyuan{第五单元}

% 散文标题,不上目录,包装的是\subsubsubsubject
\essayhead[学习目标][]

\startitemize[n,packed]
% 不归集答案的列表项使用系统的\item,或\startitem...\stopitem
    \item 按要求掌握“识字表”“写字表”中本单元的生字,理解并积累“词语表”中本单元的词语,掌握“识字表”中列出的多音字。
    \item 有感情地朗读课文,背诵并默写指定的内容。
    \item 透过语言文字想象画面,感受乡村的和谐美好。
    \item 学会向别人转述事情。
    \item 和同学分享交流你最喜欢的景致;写写自己喜欢的某个地方,表达出真实的感受。
\stopitemize

% 课文,包装的是\section
\kewen{古诗词三首}

% 栏目/题目分组,包装的是\subsubsubject(可能会改成\subsubject)
\lanmu{基础练习}

% 大题,包装的是\subsubsection
\dati{用“\backslash”划去不正确的选项。}

\startitemize[n,packed]
% 小题,包装的是\startitem...\stopitem
% 画斜线题的答案\answerS{}
    \startxiaoti 篱落(疏疏 \quad \answerS{蔬蔬})一径深,树头新绿未成(\answerS{荫} \quad 阴)。 \stopxiaoti
    \startxiaoti 日长篱落无人过,惟有蜻蜓蛱(\answerS{xiá} \quad jiá)蝶飞。 \stopxiaoti
    \startxiaoti (\answerS{矛} \quad 茅)檐低小,溪上青青草。 \stopxiaoti
    \startxiaoti 最喜小儿亡(wú \quad \answerS{wáng})赖,溪头卧剥(\answerS{bāo} \quad bō)莲蓬。 \stopxiaoti
\stopitemize

\dati{读拼音,写字词。}

% 在拼音下的田字格中写字词
% !!!字的个数多于音节时不会在题目校样中全部显示,但会在答案中呈现 TODO 校验
\pinyinge[xiǎo|háir|][小孩儿]
\pinyinge[yī|gè|màor||tóu][一个帽儿头]

\pinyinge[niǔ|kòu][纽扣]
\pinyinge[xiàng|mào][相貌]
\pinyinge[yǎng|zūn|chǔ|yōu][养尊处优]
\pinyinge[miǎo|xiǎo][渺小]

\dati{给加点的字词选择正确的意思。}

\startitemize[n,packed]
    % 着重号,下加点
    \startxiaoti 儿童急\zhuozhong{走}追黄蝶 \rBrackets{} \stopxiaoti
        \startitemize[A,horizontal,three]
            \startitem 离开 \stopitem
            % 正确选项,答案为项目编号
            \answerN{跑}
            \startitem 行走 \stopitem
        \stopitemize
    \startxiaoti 白发谁家翁\zhuozhong{媪} \rBrackets{} \stopxiaoti
        \startitemize[A,horizontal,three]
            \startitem 母亲 \stopitem
            \startitem 妇女 \stopitem
            \answerN{老妇人}
        \stopitemize
    \startxiaoti 最喜小儿\zhuozhong{亡赖} \rBrackets{} \stopxiaoti
        \startitemize[A,horizontal,three]
            \answerN{顽皮、淘气}
            \startitem 撒泼刁蛮 \stopitem
            \startitem 没有依靠 \stopitem
        \stopitemize
    \startxiaoti 四时田园杂\zhuozhong{兴} \rBrackets{} \stopxiaoti
        \startitemize[A,horizontal,three]
            \startitem 高兴 \stopitem
            \startitem 兴奋 \stopitem
            \answerN{兴致}
        \stopitemize
\stopitemize

\dati{默写《宿新市徐公店》并回答问题。}

% 居中段落
% 在下画线上填空的答案\answerU{}
{\midaligned{\answerU{宿新市徐公店}\\
\answerU{篱落疏疏一径深},\answerU{树头新绿未成阴}。\\
\answerU{儿童急走追黄蝶},\answerU{飞入菜花无处寻}。}}

这首诗的作者是 \answerU{宋} 代诗人 \answerU{杨万里},诗中运用 \answerU{动静结合} 的手法描写了 \answerU{暮春乡村} 的景色。我喜欢的诗句是 \answerU{儿童急走追黄蝶,飞入菜花无处寻},因为 \answerU{这两句诗表现了儿童的天真可爱}。

% 打钩/画钩/对钩、打叉/画叉符号
\dati{判断下列说法是否正确。(正确的画“ \TrueMark ”,错误的画“ \FalseMark ”)}

\startitemize[n,packed]
% 右侧带括号的对(\answerTR)错(\answerFR)答案
    \startxiaoti 《宿新市徐公店》描写的是暮春时节乡村的景色,而《四时田园杂兴(其二十五)》则写的是初春江南的田园景色。 \answerFR \stopxiaoti
    \startxiaoti 《四时田园杂兴(其二十五)》这首诗的前两句写景色,第三句从侧面写农民劳动的情况,最后一句用蛱蝶飞来衬托村中的寂静,静中有动,显得更静。 \answerTR \stopxiaoti
    \startxiaoti 《清平乐·村居》是宋代词人辛弃疾的词作,“清平乐”是题目,“村居”是词牌名,全词表现了词人对农村和平宁静生活的喜爱。 \answerFR \stopxiaoti
\stopitemize

\dati{阅读古诗词,完成练习。}

\essayhead[行香子·树绕村庄][秦观]

% 意大利体/斜体/着重体/楷体
{\it 树绕村庄,水满陂(bēi)塘。倚东风、豪兴徜徉。小园几许,收尽春光。
有桃花红,李花白,菜花黄。

远远苔墙,隐隐茅堂。飏(yáng)青旗、流水桥旁。偶然乘兴、步过东冈。
正莺儿啼,燕儿舞,蝶儿忙。
}

\startitemize[n,packed] % intext
    \startxiaoti 这首词的词牌名是 \answerU{行香子},我知道的词牌名还有: \answerU{卜算子}、 \answerU{沁园春}、 \answerU{念奴娇}。 \stopxiaoti
    \startxiaoti 这首词上下片完全对称,节奏明快,上下片结尾各有由一个字引领的三字排偶句(对仗句): \stopxiaoti
        有 \answerU{桃花红}, \answerU{李花白}, \answerU{菜花黄}。
        正 \answerU{莺儿啼}, \answerU{燕儿舞}, \answerU{蝶儿忙}。
    \startxiaoti 根据词句内容填空。 \stopxiaoti
    \startitemize[n,packed][left=(, right=),stopper=]
        \startxxiaoti 小园虽然很小,但也能收尽 \answerU{春光}。看那桃花 \answerU{红},李花 \answerU{白},菜花 \answerU{黄}。 \stopxxiaoti
        \startxxiaoti 远远的 \answerU{苔墙},隐隐的 \answerU{茅堂}。青色的酒旗在风中飘扬。 \stopxxiaoti
        \startxxiaoti 乘着游兴,步过 \answerU{东冈},正有莺儿 \answerU{啼},燕儿 \answerU{舞},蝶儿 \answerU{忙}。 \stopxxiaoti
    \stopitemize
\stopitemize

\lanmu{实践运用}

\dati{无论什么时候,无论什么季节,乡下人家都是一道独特的风景,一幅和谐的画卷。我们来补一补下面这些关于乡村生活的诗句吧!}

\startitemize[n,packed] % intext
    \startxiaoti \answerU{昼出耘田夜绩麻},村庄儿女各当家。 \stopxiaoti
    \startxiaoti 晨兴理荒秽, \answerU{ 戴月荷锄归}。 \stopxiaoti
    \startxiaoti 乡村四月闲人少, \answerU{才了蚕桑又插田}。 \stopxiaoti
    \startxiaoti 黄梅时节家家雨, \answerU{青草池塘处处蛙}。 \stopxiaoti
    \startxiaoti \answerU{绿树村边合},青山郭外斜。 \stopxiaoti
    \startxiaoti 水满田畴 \answerU{稻叶齐},日光穿树 \answerU{晓烟低}。 \stopxiaoti
\stopitemize


\kewen{猴王出世}

\lanmu{基础练习}

\dati{读拼音,写词语。}

\pinyinge[jié|gòu][结构]
\pinyinge[shùn|xù][顺序]
\pinyinge[hé|xié][和谐]
\pinyinge[zhuāng|shì][装饰]

\pinyinge[tǎng|ruò][倘若]
\pinyinge[shuì|mián][睡眠]
\pinyinge[sǒng|lì][耸立]
\pinyinge[zhào|lì][照例]

\dati{用“\backslash”划去下面词语中加点的字错误的读音。}

\startitemize[horizontal][columns=two,symbol=n]
    \startitem 鸡\zhuozhong{冠}花(guān \quad \answerS{guàn}) \stopitem
    \startitem \zhuozhong{朴}素(sù \quad \answerS{shù}) \stopitem
    \startitem \zhuozhong{附}会(\answerS{fǔ} \quad fù) \stopitem
    \startitem 大\zhuozhong{踏}步(\answerS{tā} \quad tà) \stopitem
\stopitemize

\dati{根据要求写词语。}

\startitemize[n,packed] % intext
    \startxiaoti 从课文中找出下列词语的反义词。 \stopxiaoti
        \startitemize[horizontal][columns=,symbol=a]
            % 在括号中填空的答案
            \startitem 朴素 — \answerB{华丽} \stopitem
            \startitem 普通 — \answerB{独特} \stopitem
        \stopitemize
    \startxiaoti 从课文中找出下列词语的近义词。 \stopxiaoti
        \startitemize[horizontal][columns=,symbol=a]
            \startitem 假设 — \answerB{倘若} \stopitem
            \startitem 协调 — \answerB{和谐} \stopitem
        \stopitemize
    \startxiaoti 在括号里填上合适的动词。 \stopxiaoti
        \startitemize[horizontal,three,a]
            \startitem \answerB{搭}瓜架 \stopitem
            \startitem \answerB{种}南瓜 \stopitem
            \startitem \answerB{翘}尾巴 \stopitem
            \startitem \answerB{爬}屋檐 \stopitem
            \startitem \answerB{找}食物 \stopitem
            \startitem \answerB{竖}旗杆 \stopitem
        \stopitemize
\stopitemize

\dati{查字典填空。}

% 缓存块整体作为答案
\startBufferAsAnswer
\startxtable[mytable]
    \startxtablehead
        \startxrow[bold]
            \startxcell 要查的字 \stopxcell
            \startxcell 音序 \stopxcell
            \startxcell 部首 \stopxcell
            \startxcell 查几画 \stopxcell
            \startxcell 在正确的解释上画“ \TrueMark ” \stopxcell
        \stopxrow
    \stopxtablehead
    \startxtablebody
        \startxrow
            \startxcell \zhuozhong{觅}食 \stopxcell
            % 根据模式确定的颜色\PTA{}(P校对色,T透明/隐藏色,A答案色)
            \startxcell \PTA{M} \stopxcell
            \startxcell \PTA{爫} \stopxcell
            \startxcell \PTA{4} \stopxcell
            % 勾选指定字符\check{},钩号用PTA色
            \startxcell[align=flushleft]\check{①}寻找 ②偷窃 ③雇 \stopxcell
        \stopxrow
        \startxrow
            \startxcell \zhuozhong{率}领 \stopxcell
            \startxcell \PTA{S} \stopxcell
            \startxcell \PTA{亠} \stopxcell
            \startxcell \PTA{9} \stopxcell
            \startxcell[align=flushleft]①大略 \check{②}带领 ③模范 ④遵循 \stopxcell
        \stopxrow
        \startxrow
            \startxcell 屋\zhuozhong{檐} \stopxcell
            \startxcell \PTA{Y} \stopxcell
            \startxcell \PTA{木} \stopxcell
            \startxcell \PTA{13} \stopxcell
            \startxcell[align=flushleft]\check{①}屋顶伸出的边沿 ②像屋檐 \stopxcell
        \stopxrow
        \startxrow
            \startxcell 和\zhuozhong{谐} \stopxcell
            \startxcell \PTA{X} \stopxcell
            \startxcell \PTA{讠} \stopxcell
            \startxcell \PTA{19} \stopxcell
            \startxcell[align=flushleft]\check{①}调和、调整 ②完结、成功 \stopxcell
        \stopxrow
    \stopxtablebody
\stopxtable
\stopbuffer\stopAsAnswer

\dati{按要求完成句子练习。}

\startitemize[n,packed] % intext
    \startitem \zhuozhong{即使}附近的石头上有妇女在捣衣,它们 \zhuozhong{也} 从不吃惊。\stopitem
        % 答案“示例”标签 \answerLE
        % 答案“略”标签 \answerLO
        用加点的词语写一句话:\answerLE\answerU{即使下雨,我们也要按时到校。}
    \startitem 你最喜欢文中的哪句话?请端端正正地抄下来,并说说你喜欢的理由。\answerLO\stopitem
        % 行内线
        句子: \thinrule

        理由: \thinrule
    \startitem 补全句子。\stopitem
        “天边的\answerU{红霞},向晚的\answerU{微风},头上飞过的\answerU{归巢的鸟儿},都是他们的好友。它们和乡下人家一起,绘成了一幅\answerU{自然}、\answerU{和谐}的田园风景画。”我觉得他们的好友还有:\answerLE\answerU{夕阳}、\answerU{大树}、\answerU{田野}。
    \startitem 辛苦了一天的人们,甜甜蜜蜜地进入梦乡。(缩句)\stopitem
        \answerU{人们进入梦乡。}
\stopitemize


\lanmu{阅读拓展}

\dati{阅读短文,完成练习。}

% 散文标题
\essayhead[姥姥家的小院][]

{\it
今年暑假,我的生活真是丰富多彩。最有趣的是到农村的姥姥家,我最喜欢姥姥家的小院。

\bz{你知道吗?写景类文章的结构一般有这样三类:总—分,分—总,总—分—总。我们要根据具体的内容进行分析。中心句大多数情况下在文章或段落的开头或结尾,起着“总起”或“总结”的作用。
}

院子西南角,几根长竹竿架上,爬满了花藤,稠密的绿叶衬着紫红色的花朵,又娇嫩,又鲜艳,远远望去,好像一匹美丽的花布。

院子的东北角是由金瓜架构成的棚子。光滑的金瓜像吊灯似的挂在藤上。瓜棚下面放着桌子。晚上,灯亮了,孩子们在这里读书、下棋,大人们也在这里乘凉、闲谈。

院子东面栽着几株像巨人一样高大的白杨,还有几株果实累累的果树。白杨树挺直了高大的躯干,碧绿的叶子被风吹得哗哗直响。空气中弥漫着果子的清香。

院子南面是一片碧绿的小菜园。韭菜绿油油的,茄秧上挂满了嫩茄子,半红半绿的辣椒像害羞似的,藏在茂密的绿叶中,还有西红柿、黄瓜……各种各样的蔬菜,为农家增添了乐趣。

姥姥家的小院,多么富有诗情画意啊!
}

\startitemize[n,packed] % intext
    \startxiaoti 短文的结构为\answerU{总—分—总},全文是围绕哪一句话写的?请把这句话抄下来。\stopxiaoti
        \answerU{姥姥家的小院,多么富有诗情画意啊!}
    \startxiaoti 按要求填空。\stopxiaoti
姥姥家的院子西南角有\answerU{爬满了花藤的竹架},东北角是\answerU{金瓜架构成的棚子},院子东面栽着\answerU{白杨树},院子南面是\answerU{小菜园},姥姥家的小院多么富有诗情画意啊!
    \startxiaoti 这篇文章是按\answerU{方位}顺序来写的,\answerU{西南角}、\answerU{东北角}、\answerU{东面}、\answerU{南面}这几个词语体现了这种顺序。 \stopxiaoti
\stopitemize


\lanmu{实践运用}

\dati{《乡下人家》交叉采用了“植物—动物—人物”的对象顺序,“房前—屋后—河中”的空间顺序,“春—夏—秋”的季节顺序,“白天—傍晚—夜间”的时间顺序,运用比喻、拟人等修辞手法为我们描绘了“瓜藤攀檐图”“鲜花绽放图”“雨后春笋图”“鸡鸭觅食图”“院落晚霞图”“秋夜睡梦图”六幅自然和谐的田园风景画,全文运用了“分—总”的结构。请你按照一定的顺序,运用“分—总”的结构描写一段你喜欢的景致,注意要写出“画面感”。}

\thinrules[n=4]\answerLO

\kewen{祖父的园子}

\dati{读拼音,写词语。}

\pinyinge[wēi|lì][威力]
\pinyinge[shǎn|shuò][闪烁]
\pinyinge[ruì|lì][锐利]
\pinyinge[sǎo|dàng][扫荡]
\pinyinge[qí|huàn][奇幻]

\pinyinge[fù|zá][复杂]
\pinyinge[māo|tóu|yīng][猫头鹰]
\pinyinge[bà|qì|shí|zú][霸气十足]

\dati{用“\backslash”划去下面词语中加点的字错误的读音。}

% 用\hfil分组两段对齐
暖\zhuozhong{和}(\answerS{hé} \quad huo) \hfil
一\zhuozhong{瞥}(piē \quad \answerS{piě}) \hfil
慰\zhuozhong{藉}(jiè \quad jí)

\zhuozhong{卜}落(bǔ \quad \answerS{bo}) \hfil
\zhuozhong{露}天(lù \quad \answerS{lòu}) \hfil
玻\zhuozhong{璃}(\answerS{lǐ} \quad li)

\dati{比一比,再组词。}

% 带括号的多行组(手动分行文本块),指定文本宽度em和括号倍数
\kuohaozu[7,2]{%
    帐\answerB{帐篷}\\
    账\answerB{结账}}
\kuohaozu[7,4]{%
    摊\answerB{摆摊}\\
    滩\answerB{沙滩}\\
    傩\answerB{傩戏}}
\kuohaozu[7,2]{%
    蝠\answerB{蝙蝠}\\
    福\answerB{福气}}
\kuohaozu[7,2]{%
    蝙\answerB{蝙蝠}\\
    编\answerB{编织}}

\dati{按要求写词语。}

\startitemize[n,packed] % intext
\startxiaoti 从文中找出下列词语的反义词。 \stopxiaoti

寒冷 \answerB{暖和} \hfil
无   \answerB{有} \hfil
简单 \answerB{复杂} \hfil
虚   \answerB{实}

\startxiaoti 填上合适的量词。\answerLE \stopxiaoti

一\answerB{扇}木板窗 \quad
一\answerB{个}小方洞 \quad
一\answerB{幅}风景画

一\answerB{道}黑影 \qquad
一\answerB{片}空白 \qquad
一\answerB{缕}炊烟

\stopitemize

\dati{根据课文内容填空。}

\startitemize[n,packed] % intext
\startxiaoti 这篇文章的作者是现代著名作家 \answerU{茅盾} 。 \stopxiaoti
\startxiaoti 我认为这是一扇 \answerU{神奇} 的天窗,因为 \answerU{它打开了孩子们神奇的想象力和好奇心}。 \stopxiaoti
\startxiaoti 最初,人们装天窗是为了 \answerU{采光} 、 \answerU{取暖} ,没想到在黑洞似的屋里,小小的天窗却成了孩子们的 \answerU{精神慰藉} 。文中的“天窗”象征了 \answerU{儿童通往想象世界的窗口} 。 \stopxiaoti
\stopitemize

\lanmu{阅读拓展}

\dati{阅读短文,完成练习。}

\essayhead[窗外][]

\answerLI

{\it
每当空闲时,我总习惯站在窗户前看看窗外。透过那一层厚而明亮的玻璃,
我时常能够看到人们忙碌的身影,听到象征着劳动的喧哗声。

早上,晨曦洒进窗户,又是全新的一天。有时,天空中可能还会有一点
儿朦朦(\answerCU{胧胧} \quad 笼笼)的雾,给人一种置身仙境的感觉。透过雾,能够看到
许多学生骑着自行车或电动车向校园飞驰而去。有时还能够遇到几个小懒虫,
上学的大部队都已经走远了,他们才慢悠悠地出发。窗户外面,恰好有几株
高大的绿化树,所以清晨在窗户前听到几声鸟鸣,是再正常不过的事了。虽
然大多数时间都是最常听见的麻雀声,但在城市的早晨,这声音也是弥足珍
贵的。隔着玻璃看着在枝条上跳跃的麻雀,我明白,鸟儿们也已经起床了,
\answerCU{这难道不正好验证了那句俗语——“早起的鸟儿有虫吃”吗?}人们忙碌,鸟
儿欢叫,构成了一幅无比和(协 \quad \answerCU{谐})的画卷。

\bz{
反问句的特点:只问不答,答案含在句中。

反问句比陈述句语气强烈得多。

陈述句改反问句:一加反问词;二改肯定或否定词;三改标点符号;四查与原句意思是否相同。
}

午后,太阳将炎热的阳光投射到人间,也投射进我的窗户,窗户外的知
了也在尽情地享受着这夏(\answerCU{末} \quad 未)的午后,忍不住高歌一曲!也许对这些
爱唱歌的精灵们来说,已是时日不多,这鸣叫也似乎在告(戒 \quad \answerCU{诫})窗前的我:
“一寸光阴一寸金”,莫虚度。

傍晚的窗外,大街上人来人往,有三五成群刚从校园回来的学生;有一
脸疲意匆匆忙忙刚下班的上班族;有满脸洋溢着喜悦的菜贩,看来今天又多
赚了三五块!远处的小广场上,已经有大妈们开始摆弄跳健身操的音箱了。
人们都在享受这一段悠闲的时间,打算将一天的压力,在这个傍晚释放,在
这一天的最后,为自己做一个完美的收尾!

一个普通的窗户,几块也许你从来不会多看几眼的玻璃,却成了一张无
形的画布,见证着一天中分分秒秒的变化,展示着动态的生活画卷。
}

\startitemize[n,packed] % intext
\startxiaoti 用“\hl[2]”从文中括号里选出正确的字。\answerLIA \stopxiaoti
\startxiaoti 仿写词语。\answerLE \stopxiaoti

慢悠悠 \qquad \answerU{胖乎乎} \quad \answerU{绿油油} \quad \answerU{美滋滋} 

人来人往 \quad \answerU{毛手毛脚} \quad \answerU{自由自在} \quad \answerU{百发百中}

\startxiaoti 文中有一句关于珍惜时间的名言:\answerU{一寸光阴一寸金}。你还能写出一句这样的名言吗?试一试:\answerLE\answerU{少壮不努力,老大徒伤悲。} \stopxiaoti
\startxiaoti 用“\hl[2]”画出文中的反问句,并把它改成陈述句。\answerLIA\answerU{这正好验证了那句俗语——“早起的鸟儿有虫吃”。} \stopxiaoti
\item这篇文章是按\answerU{时间}顺序来写的,作者透过那一层厚而明亮的玻璃窗所看到的景象,写出了老百姓\answerU{忙碌与幸福}日常生活的。

\stopitemize

\lanmu{实践运用}

\dati{透过那小小的玻璃天窗,你可能还会想象到无数神奇的东西……请你发挥想象,用具体、生动的文字把你想象到的某一画面写下来。}

\thinrules[n=4]\answerLO

\stoptext
