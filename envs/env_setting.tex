%%%% 全局配置
\startenvironment

% 校对模式
% yes no keep(保持预先设置,其后调整无效,可在命令行设置)
\definemode[proofread][keep]
\enablemode[proofread]
% \disablemode[proofread]
\definemode[answersheet][keep]
\disablemode[answersheet]

%%%% 纸张和版面
\definepapersize[sheet][width=145mm, height=210mm]
% Create two A4-derived paper sizes
\definepapersize[standing][A4][sheet]
\setuppapersize[standing]
\setuplayout[
    marking=on, % 裁切线,设置较大幅面打印纸才能显示
    location=middle,
]

% 中文配置
\mainlanguage[cn]
\language[cn]
\setscript[hanzi] % 汉字处理脚本(断行)
\setupalign[hanging,hz] %行末标点悬挂

% 汉字字体配置
\usetypescriptfile[mscore]
\usebodyfont   [mschinese,14pt]
% \usebodyfont   [mschinese-light,12pt]
% \setupbodyfont [mschinese-literate,12pt]
% \usebodyfont   [mschinese-literate,12pt]
% \definebodyfontenvironment[24pt]
% \definebodyfontenvironment[18pt]
% 定义字体
% \definefont[kaiti][name:kaiti*default at 24pt]
\definefont[checkMarkFont][name:dejavusansmono*default]
\define\TrueMark{{\checkMarkFont ✔}}
\define\FalseMark{{\checkMarkFont ✗}}

% 颜色
\setupcolors[state=start] % 开启颜色(默认关闭,转成灰度)
% !!!如果不设置透明度,用于下画线时,字色透明度会覆盖下画线的透明度
\definecolor[secondColor]     [c=1,t=1,a=1] % 双色中的彩色
\definecolor[aboveColor]      [c=1,t=0.8,a=1] % 双色中的彩色+透明(用在文字上方)
\definecolor[proofColor]      [m=1,t=0.8,a=1] % 校对文本色+透明
\definecolor[transparentColor][k=0,t=0,a=1] % 透明色(隐藏文本用)

% 结构转换集合(序号集合)
% \definecounter[subsubsectioncounter][way=bysection] % 每section重起
% \defineconversion[cnbysection][\convertnumber{cn}{\rawcountervalue[subsubsectioncounter]}]
\definestructureconversionset[cnconversion]
    [R,   % part: Rome
    cn,  % chapter 章/单元 中文数字
    ,   % section 节/课文标题
    ,   % subsection 小节/分组标题 
    cn,   % subsubsection 小小节/大习题 中文数字
    n]   % subsubsubsection 小小小节/小习题 印度数码
    [r] % Default setting

% 标题统一设置(个别设置后面单独重设)
% \defineresetset[myresetset][1,1,1,1,0][1] %关闭第3级标题的序号重置(全局设置,局部处理通过defineconversion)
% \defineresetset[subsubsectionresetset][][0] %关闭,用于单个标题的设置
\setupheads
    [part,chapter,section,subsection,subsubsection,subsubsubsection]
    [%
    after=, % 覆盖上下行插入的空格
    sectionconversionset=cnconversion,%不要混入空格
    number=no, % 不显示序号
    indentnext=yes, % 其后首行缩进
    sectionsegments=current, % 序号段:只包括当前序号
    % sectionresetset=myresetset, %
    ]
\setupheads
    [subsection,subsubsection,subsubsubsection]
    [%
    before=, % 覆盖上下行插入的空格
    ]

% 配置标题、目录格式
% 单元,新页起 , page=yes新页/left偶数页/right奇数页(默认)
\setuphead[chapter][page=yes,style={\setupalign[middle]\tfb}]
% 课目
\setuphead[section][style={\setupalign[middle]\tfa}]
% 分组/栏目标题
\setuphead[subsection][style={\tf}, color=secondColor, number=yes,distance=0pt]
% 无序号大题
\setuphead[subsubsubject][style={\bf}]
% 大题
\setuplabeltext[cn][subsubsection={{},{、}}] % 前后标签
\setuphead[subsubsection][style={\tf}, color=secondColor, number=yes,distance=0pt,]

% 自订文章标题(6级)、作者(7级)
\definehead[centersection][subsubsubsubject][align=middle, before=, after=,]
\definehead[authorsection][subsubsubsubsubject][align=middle, before=, after=,]
% \definehead[leftsection][subsubsubject][align=flushleft]
\def\essayhead[#1][#2]{%
    \centersection{#1}%
    \doifnotempty{#2}{\authorsection{#2}}%
    }
% \define[2]\rhymehead{\leftsection{#1}\leftaligned{\itx #2}\blank[0cm]}
% \define[2]\classicalpoemhead{\centersection{#1}\middlealigned{\itx #2}\blank[0cm]}

% 短文标题
\def\centralTitle[#1][#2]{%
    \subsubsubject{\tf }%
}

\setupheadtext[content=目录]
\setupcombinedlist[content][list={chapter,section},alternative=c]%
\setuplist[section][width=10mm, style=bold]
\setuplist[subsection][width=20mm]
\setuplist[subsubsection][width=20mm, style=slanted, pagestyle=normal]
% pdf交互/链接
\setupinteraction[state=start,color=,]
% pdf书签
\setupinteractionscreen[option={bookmark}]
\placebookmarks[chapter, section, subsubsection] % 可加入自定义\definelist[mylist]

% 设置页码格式
\setuppagenumbering[
    alternative=doublesided,
    location={header, margin},
]

% 段落间距
\setupwhitespace[none]
% 条目间距(覆盖预设)
% \setupitemgroup[packed]
% 行距设置
\setupinterlinespace[line=1.5em]
% 缩进设置
\setupindenting[yes, 2em, first]
% 窄行、缩进设置(每一级的缩进量)
\setupnarrower[left=01em]%
\setupcolumns[n=2, separator=rule]


% % 引用、交互
% \setupreferencing[left=,right=] % 覆盖左右两侧的引号
% \setupinteraction   [state=start, color=,contrastcolor=black,style=]
% \define[1]\see{\at{}{页}[#1]\about[#1]}% 另见

% % 页眉
% % \setupheadertexts[{\getmarking[subsubject]}][][][{\getmarking[subsubject]}]
% \setupheadertexts%
%     []%
%     [{\ReadFile{header.lua}}]%
%     [{\ReadFile{header.lua}}]%
%     []%

%%%%%%%%%%%%% 使用模块(保持顺序) %%%%%%%%%%%%%
% 竖排
% \usemodule[vtypeset]
% 
% 标点压缩与支持
\usemodule[zhpunc][pattern=kaiming, spacequad=0.5, hangjian=false]
% 
% 四种标点压缩方案:全角、开明、半角、原样:
%   pattern: quanjiao(default), kaiming, banjiao, yuanyang
% 行间标点(转换`、,。.:!;?`到行间,pattern建议用banjiao):
%   hangjian: false(default), true
% 加空宽度(角):
%   spacequad: 0.5(default)
% 
% 行间书名号和专名号(\bar实例):
%   \zhuanmh{专名}
%   \shumh{书名}
% 
% 夹注
% \usemodule[jiazhu][fontname=tf, fontsize=10.5pt, interlinespace=0.2em]
% default: fontname=tf, fontsize=10.5pt, interlinespace=0.08em(行间标点时约0.2em)
% fontname和fontsize与\switchtobodyfont的对应参数一致
% 夹注命令:
%   \jiazh{夹注}
%%%%%%%%%%%%%%%%%%%%%%%%%%%%%%%%%%%%%%%%%%%%%%%%%%%%

% 自动名称(用于自动生成宏、引用等)
\definecounter[uniquecounter][way=bytext]
% \setnumber[uniquecounter][1]
\def\incrementuniquename{\incrementcounter[uniquecounter]}
\def\uniquename{uniquename\Characters{\rawcountervalue[uniquecounter]}}%

% \someheadnumber[subsubsection][current]
% %%%% 存取数据
\definedataset[answertable][delay=yes] % 最后实现,会插入页码信息???
\def\saveAnswer[#1]#2{%
    \doifnotmode{answersheet}{%
        % \incrementuniquename%
        % \expandafter\def\csname \strippedcsname\uniquename\endcsname{\someheadnumber[subsubsection][current]}%
        \setdataset[answertable][cmd={#1}, answer={#2}]%花括号可以避免数据中的符号引起错误
        }%
}

% 着重号 TODO 居中
\definebar[zhuozhong][text=\lower0.6em\hbox{\hskip0.65em ·}, repeat=no]

% 条件文本:校对标记红P、校对标记透明T、答案不着色A
% 
\def\updatePTAColor{%
    \doifmodeelse{answersheet}%
    {\definecolor[PTAColor][aboveColor]}% default 默认颜色
    {\doifmodeelse{proofread}%
        {\definecolor[PTAColor][proofColor]}%校对色
        {\definecolor[PTAColor][transparentColor]}%常态透明
    }%
}
\def\PTA#1{%
    \updatePTAColor%
    \color[PTAColor]{#1}%
}

% 原样输出
\def\answerAsItIs#1{\hskip0pt #1 \hskip0pt} % 有空格,否则插入的quad无效 TODO

% 排印缓存并保存为答案
\def\startBufferAsAnswer{
    \incrementuniquename
    \startbuffer[\uniquename]}
\def\bufferAsAnswer#1{\par\page[no]#1}%禁止分页无效 TODO
\def\stopAsAnswer{
    \getbuffer[\uniquename]
    \saveAnswer[bufferAsAnswer]{\getbuffer[\uniquename]}}
% 因为\startbuffer开始后直至\stopbuffer都被认为是buffer,所以\stopbuffer无法包装
% 结束命令为:\stopbuffer\stopAsAnswer
% TDO 也许有控制方法,如此则可以definestartstop,或自定起讫命令

% 自定义标题
% 单元
\def\danyuananswer#1{\section{#1}}
\def\danyuan#1{%
    \chapter{#1}%
    \saveAnswer[danyuananswer]{#1}%
}
% 课文标题
\def\kewenanswer#1{\blank\subsubsubject{#1}}
\def\kewen#1{%
    \section{#1}%
    \saveAnswer[kewenanswer]{#1}%
}
% 大题目
% TODO 
\def\datiformat#1{\par{\secondColor{#1}、}}
\def\dati#1{%
    \incrementuniquename%
    \subsubsection[\uniquename]{#1}%
    \saveAnswer[datiformat]{\structurenumber}%获取实际的值
    % 获取计数器的引用,必须在编号标题中或其后后才能显示,如\subsubsubject{#1、}
    % \saveAnswer[datiformat]{\ref[number][\uniquename]}%
}
% 小题(一级列表项),用法:\startxiaoti \stopxiaoti
\def\xiaotiformat#1{{\secondColor{#1}.\nobreak}}
% 使用before={},比直接将\incrementuniquename插在行头更稳健(会受到周围空格的影响)
\definestartstop[xiaoti][
   before={%
        \incrementuniquename%
        \startitem[\uniquename]%
        \saveAnswer[xiaotiformat]{\ref[number][\uniquename]}},
   after={\stopitem}]
% \def\xiaoti{\item \saveAnswer[xiaotiformat]{\currentstructurelistnumber}} %数字错误

% 小小题,二级列表项,用法:\startxxiaoti \stopxxiaoti
\def\xiaoxiaotiformat#1{{\secondColor{(#1)\nobreak}}}
\definestartstop[xxiaoti][
   before={%
        \incrementuniquename%
        \startitem[\uniquename]%
        \saveAnswer[xiaoxiaotiformat]{\ref[number][\uniquename]}},
   after={\stopitem}]

% 下画线颜色
\setupthinrules[color=secondColor]
% 下画线完形填空answerUnderbar
% !!!标点符号处理过程导致下画线断开,或禁则失效
% 可能是标点模块删除标点前后的空白造成的
% 临时处理方法是只在填空前面加kern
\definebar[clozeBar][underbar][continue=yes, rulethickness=0.25pt, color=secondColor, offset=-0.5]
\def\answerU#1{%
    \startbar[clozeBar]%
    \color[transparentColor]{#1}\PTA{#1}% 
    % 下面的方法会导致断句困难
    % \scratchdimen\widthofstring{#1}\relax % 获取文本宽度
    % \zwj\kern1\scratchdimen \color[black]{#1}% \kern.5\scratchdimen\zwj % U+200D ZERO WIDTH JOINER
    \stopbar%
    \saveAnswer[answerAsItIs]{#1}%
}%

% 回答画掉slashout
\def\answerS#1{%
    \setbox0=\hbox{#1}%
    % 阻止切换到v模式以免断行
    \dontleavehmode\copy0\kern-\wd0\hbox to \wd0{\hfil\PTA{\backslash}\hfil}%
    \saveAnswer[answerS]{#1}%
}

% 回答勾选check
\def\check#1{%
    \setbox0=\hbox{#1}%
    % 阻止切换到v模式以免断行
    \dontleavehmode\copy0\kern-\wd0\hbox to \wd0{\hfil\PTA{\TrueMark}\hfil}%
}
\def\answerC#1{%
    \check{#1}%
    \saveAnswer[answerC]{#1}%
}

% 回答画选checkUnderbar
\definebar[checkBar][clozeBar][color=PTAColor]
\def\checkU#1{%
    \startbar[checkBar]%
    #1% 
    \stopbar%
}
\def\answerCU#1{%
    \updatePTAColor%
    \checkU{#1}%
    \saveAnswer[answerCUformat]{#1}%
}
\def\answerCUformat#1{\hskip0pt \answerCU{#1} \hskip0pt}

% 回答序号
\def\answerN#1{%
    \incrementuniquename%
    \startitem[\uniquename]#1%
    \raise 1em\vbox to 0em{\hbox to 0em{\itx\PTA{\ref[number][\uniquename]}}}%
    \stopitem%
    \saveAnswer[answerAsItIs]{\ref[number][\uniquename]}%
}

% TODO 列表
% \saveAnswer[answerAsItIs]{\currentstructurelistnumber}%

% 回答“略” omit
\def\LableO{(略)}
\def\answerLO{%
    \raise 1em\vbox to 0em{\hbox to 0em{\itx\PTA{\LableO}}}%
    \saveAnswer[answerAsItIs]{\LableO}%
}

% 回答“示例:”example
\def\LableE{示例:}
\def\answerLE{%
    \raise 1em\vbox to 0em{\hbox to 0em{\itx\PTA{\LableE}}}%
    \saveAnswer[answerAsItIs]{\LableE}%
}

% 回答“文中练习”inline TODO 合并多题的提示
\def\LableI{【文中练习】}
\def\answerLI{%
    \dontleavehmode%
    \raise 1em\vbox to 0em{\hbox to 0em{\itx\PTA{\LableI}}}%
    \saveAnswer[answerAsItIs]{\LableI}%
}

% 回答“见文中练习”inline as given above
\def\LableIA{(见\LableI)}
\def\answerLIA{%
    \raise 1em\vbox to 0em{\hbox to 0em{\itx\PTA{\LableIA}}}%
    \saveAnswer[answerAsItIs]{\LableIA}%
}

% 括号
% TODO 测量宽度;允许输入宽度
\def\brackets#1{(\enskip \PTA{#1} \enskip)}
% 括号(行末)
\def\rBrackets#1{\hfill \brackets{#1}}

% 括号中画勾
\def\answerT{%
    \brackets{\TrueMark}%
    \saveAnswer[answerAsItIs]{\TrueMark}%
}
% 括号(行末)中画勾
\def\answerTR{%
    \rBrackets{\TrueMark}%
    \saveAnswer[answerAsItIs]{\TrueMark}%
}

% 括号中画叉
\def\answerF{%
    \brackets{\FalseMark}%
    \saveAnswer[answerAsItIs]{\FalseMark}%
}
% 括号(行末)中画叉
\def\answerFR{%
    \rBrackets{\FalseMark}%
    \saveAnswer[answerAsItIs]{\FalseMark}%
}

% 填入括号
\def\answerB#1{%
    \brackets{#1}%
    \saveAnswer[answerAsItIs]{#1}%
}
% 填入括号(行末)
\def\answerBR#1{%
    \rBrackets{#1}%
    \saveAnswer[answerAsItIs]{#1}%
}

% 田字格
% 导入lua模块MPwordslink.lua
\startluacode
    require "envs/Pinyinge"
\stopluacode
\def\pinyinge[#1][#2]{
    \dontleavehmode
    \startMPcode
        lua.MP.tianzige("#1","#2") ;
    \stopMPcode
    \saveAnswer[answerAsItIs]{#2}
}

% 边注
\defineframedtext[bianzhu]
    [corner=round,
    width=12.5em,
    indenting={yes, 2em, first}, % 无效
    offset=0.25em, % 内容向内偏置(可用loffset等分别指定)
    frameoffset=0em, % 外框向外偏置
    location=hanging, % depth, low, hanging, none
    style={\it}]
\definefloat[bianzhumezzo][bianzhumezzos][intermezzo]
\setupcaption[bianzhumezzo][location=none] % 不加标题
\define[1]\bz{%
    \placefloat[bianzhumezzo][right,nonumber]{不显示标题}{\bianzhu{#1}}}

% 编号段落
\newcounter\Paracount
\def\Paragraphnumber%
    {\increment\Paracount 
    \Paracount.~}
\def\startParagraphNumbers%
    {\par \begingroup \appendtoks \Paragraphnumber \to \everypar}
\def\stopParagraphNumbers%
    {\par \endgroup}




% 括号分组
% TODO 调整前头的空间;自动计算高度;绘制大括号
% vbox式只能用分栏的形式并排,以获取vbox的宽度
% 换成图形则可用combination并排
\def\kuohaozu#1{%
    % 用lua逐行处理,未完成
    % \startluacode
    %     MP = MP or {}
    %     % MP.lines = string.splitlines([=[#1]=]) --无效
    %     % MP.lines = string.explode([=[#1]=], [=[\\]=]) --乱
    %     MP.lines = string.split([=[#1]=], [=[\\]=])
    % \stopluacode
    % 
    % \startMPcode
    % draw textext.lft("\[") yscaled 2;
    % for i=1 upto lua("mp.size(MP.lines)"):
    %     draw textext.rt(lua("mp.quoted(MP.lines[" & decimal i & "])"));
    % endfor ;
    % \stopMPcode
    \dontleavehmode%
    \startMPcode
    % 用加框文本可以实现更复杂的控制
    % draw textext.lft("\[") yscaled 2;
    % draw textext.rt("\framedtext[%
    %     offset=default,%
    %     location=depth,%
    %     frame=off,%
    %     % width=7em,% local fit max broad fixed dimension
    %     % autowidth=force,% yes no force 无效
    %     ]{#1}");
    % 
    % 简易方式
    draw textext.lft("\strut \[") yscaled 2;
    draw textext.rt("\vbox{#1}")
    \stopMPcode
}

% 表格
\definextable[mytable][offset=0.2em, option={stretch, width}, align={center,lohi}]

\replacemissingcharacters % 缺字替换

\stopenvironment