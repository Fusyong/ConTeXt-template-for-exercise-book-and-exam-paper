%%%% 全局配置
\startenvironment % 与第一行代码间必须有空行 BUG

% %%%%%%%%%%%%%%%%%%%%%%%%%%%%%%%%%%%%%%%%%%%%%%
% 模式
% %%%%%%%%%%%%%%%%%%%%%%%%%%%%%%%%%%%%%%%%%%%%%%
% 校对模式,yes no keep(保持预先设置,其后调整无效,可在命令行设置)
\definemode[proofread][keep]
\enablemode[proofread]
% 答案模式,请在答案部分开启
\definemode[answersheet][keep]
\disablemode[answersheet]

% %%%%%%%%%%%%%%%%%%%%%%%%%%%%%%%%%%%%%%%%%%%%%%
% 设置变量 TODO
% %%%%%%%%%%%%%%%%%%%%%%%%%%%%%%%%%%%%%%%%%%%%%%
\setvariables[settings][%
    % 如果紧跟在startenvironment后,上行末尾需要注释,
    % 且注释前不能有空格,否则会引入空白页
    smallblank=quarterline,% 板块间的最小间距
    bodyfont=14pt,%正文字号 A5a类21—16,b类14,c类12,d类10.5
    interlinespace=1.5em,
]
\def\smallblank{\blank[\getvariable{settings}{smallblank}]}
% 此处samepage在某些地方不起效,而在当地手动加则有效 TODO
\def\smallblanksamepage{\blank[\getvariable{settings}{smallblank},samepage]}

% %%%%%%%%%%%%%%%%%%%%%%%%%%%%%%%%%%%%%%%%%%%%%%
% 纸张、版面、页码、页眉
% %%%%%%%%%%%%%%%%%%%%%%%%%%%%%%%%%%%%%%%%%%%%%%
\setuppapersize[A4][A4,oversized] % A4版面,A4纸+oversized(1.5cm)打印(扩展以便看到裁切线)
\definemeasure[Bleed][3mm]%出血
\definemeasure[Trim][7.5mm]%剪裁线偏置等于oversized
\setuplayout[%
    backspace=20mm,% 内侧空白
    % 版心width,排中文时要取正文字号的整数倍,即`每行字数*字号`
    % A4正常可用width约160mm,约合14pt*33字*0.35146mm/pt=162.4mm
    % A5a类106*168,b类108*167,c类106*164,d类107*166
    width={\dimexpr\getvariable{settings}{bodyfont}*33\relax},
    % 或直取M宽/方空铅的整数倍,即`每行字数em`(与上法有极小的差别,与字体有关)
    % A5a类19*20,b类22*22,c类25*25,d类29*28
    % width=33em,
    leftmargin=0em,
    rightmargin=4em,
    topspace=20mm,
    height=250mm,
        header=0mm,
        footer=7.5mm,
    marking=on, % 裁切线,打印纸设置大于版面时才能显示
    location=middle, % 居中放置在纸张上
    bleedoffset=\measure{Bleed},
    trimoffset=-\measure{Trim},
]
% 页码
% !!!alternative=doublesided指定对页交替
% (一般setup­layout设置默认指奇数页,对页交替使偶数页采用奇数页的镜像值)
\setuppagenumbering[alternative=doublesided,location={footer,right}]
% 文前用小写罗马字
\definestructureconversionset [frontpart:pagenumber][][r]
\definestructureconversionset [bodypart:pagenumber][][n]
\startsectionblockenvironment [bodypart]
    \setcounter[userpage][1]%重排正文页码
\stopsectionblockenvironment

% 页眉
% % \setupheadertexts[{\getmarking[subsubject]}][][][{\getmarking[subsubject]}]
% \setupheadertexts%
%     []%
%     [{\ReadFile{header.lua}}]%
%     [{\ReadFile{header.lua}}]%
%     []%

% %%%%%%%%%%%%%%%%%%%%%%%%%%%%%%%%%%%%%%%%%%%%%%
% 中文配置
% %%%%%%%%%%%%%%%%%%%%%%%%%%%%%%%%%%%%%%%%%%%%%%
\mainlanguage[cn]
\language[cn]
\setscript[hanzi] % 汉字处理脚本(断行等)
\setupalign[hanging,hz] %行末标点悬挂

% %%%%%%%%%%%%%%%%%%%%%%%%%%%%%%%%%%%%%%%%%%%%%%
% 字体和符号配置
% %%%%%%%%%%%%%%%%%%%%%%%%%%%%%%%%%%%%%%%%%%%%%%
\input envs/type-imp-student.mkiv
\usetypescriptfile[mscore]%使用系统自带的字体脚本
\setupbodyfont   [mschinese,\getvariable{settings}{bodyfont}]

% 定义画勾画叉符号及字体
\definefont[checkMarkFont][name:dejavusansmono*default]
\define\TrueMark{{\checkMarkFont ✔}}
\define\FalseMark{{\checkMarkFont ✗}}
% 着重号 TODO 居中
\definebar[zhuozhong][text=\lower0.4em\hbox{\hskip0.45em .}, repeat=no]

% %%%%%%%%%%%%%%%%%%%%%%%%%%%%%%%%%%%%%%%%%%%%%%
% 设置颜色
% %%%%%%%%%%%%%%%%%%%%%%%%%%%%%%%%%%%%%%%%%%%%%%
\setupcolors[state=start] % 开启颜色(默认关闭,转成灰度)
% !!!如果不设置透明度,用于下画线时,字色透明度会覆盖下画线的透明度
\definecolor[secondColor]     [c=1,t=1,a=1] % 第二色(青色)
\definecolor[aboveColor]      [c=1,t=0.8,a=1] % 上层色(80%透明的青色)
\definecolor[lightSecondColor][c=0.3,t=1,a=1] % 浅二色(青色)
\definecolor[proofColor]      [m=1,t=0.8,a=1] % 校对文本色(80%透明的品色)
\definecolor[transparentColor][k=0,t=0,a=1] % 透明色(就地隐藏文本用)

% %%%%%%%%%%%%%%%%%%%%%%%%%%%%%%%%%%%%%%%%%%%%%%
% 结构编号转换集合(各级标题编号的呈现方式)
% %%%%%%%%%%%%%%%%%%%%%%%%%%%%%%%%%%%%%%%%%%%%%%
% \definecounter[subsubsectioncounter][way=bysection] % 每section重起
% \defineconversion[cnbysection][\convertnumber{cn}{\rawcountervalue[subsubsectioncounter]}]
\definestructureconversionset[cnconversion]
    [R,  % part                部分,大写罗马数字
    cn,  % chapter             章/单元 中文数字
    ,    % section             节/课文标题
    ,    % subsection          小节/分组标题 
    cn,  % subsubsection       大题 中文数字
    n,   % subsubsubsection    小题 印度数码
    n,]   % subsubsubsubsection 小小题 印度数码
    [r] % Default setting 小写罗马字

% %%%%%%%%%%%%%%%%%%%%%%%%%%%%%%%%%%%%%%%%%%%%%%
% 设置标题
% %%%%%%%%%%%%%%%%%%%%%%%%%%%%%%%%%%%%%%%%%%%%%%
% 统一设置(个别标题后面可单独重设)
% \defineresetset[myresetset][1,1,1,1,0][1] %关闭第3级标题的序号重置
% \defineresetset[subsubsectionresetset][][0] %关闭,用于单个标题的设置
\setupheads
    [part,chapter,section,subsection,subsubsection,subsubsubsection]
    [%
    after=, % 覆盖其后插入的空
    sectionconversionset=cnconversion,%不要混入空格
    number=no, % 不显示序号
    indentnext=yes, % 其后首行缩进
    sectionsegments=current, % 序号段:只包括当前序号
    % sectionresetset=myresetset, % 序号重置集合
    ]
% 部分重设
\setupheads
    [subsection,subsubsection,subsubsubsection]
    [%
    before=, % 覆盖其前插入的空
    ]
% 单元设置,新页起 , page=yes新页/left偶数页/right奇数页(默认)
\setuphead[chapter][page=yes,style={\setupalign[middle]\tfb}]
% 课文标题设置
\setuphead[section][style={\setupalign[middle]\tfa}]
% 分组/栏目标题设置
\setuphead[subsection][style={\bf}, color=secondColor, number=no,distance=0pt]
% 大题设置
\setuplabeltext[cn][subsubsection={{},{、}}] % 前后标签
%after={\blank[samepage]}无效,受到dati后续命令的干扰
\setuphead[subsubsection][style={\tf}, color=secondColor, number=yes,distance=0pt,]
% 自订文章标题(6级)、作者(7级)
\definehead[centersection][subsubsubsubject][align=middle, before=, after=,]
\definehead[authorsection][subsubsubsubsubject][align=middle, before=, after=,]
% \definehead[leftsection][subsubsubject][align=flushleft]
\def\essayhead[#1][#2]{\centersection{#1}%
    \doifnotempty{#2}{\authorsection{#2}}%
    }
% \define[2]\rhymehead{\leftsection{#1}\leftaligned{\itx #2}\blank[0cm]}
% \define[2]\classicalpoemhead{\centersection{#1}\middlealigned{\itx #2}\blank[0cm]}
% 短文标题
\def\centralTitle[#1][#2]{\subsubsubject{\tf }}

% 设置列表组itemize、所有级别、阿拉伯数字、堆叠、其后不留空
\setupitemgroup[itemize][each][n,packed][before={\smallblank}, after={\smallblank},]
% 设置二级列表
\setupitemgroup[itemize][2][left=(, right=),stopper=]
% 定义答案列表,大写字母序号、横排、三列
\defineitemgroup[answeritems]
\setupitemgroup[answeritems][each][A,horizontal,three]

% %%%%%%%%%%%%%%%%%%%%%%%%%%%%%%%%%%%%%%%%%%%%%%
% 设置目录、PDF书签、交互、引用
% %%%%%%%%%%%%%%%%%%%%%%%%%%%%%%%%%%%%%%%%%%%%%%
\setupheadtext[content=目录]
\setupcombinedlist[content][list={DANYUAN,KEWEN,lanmuN},alternative=c]%
\setuplist[DANYUAN][style=bold]
\setuplist[KEWEN][margin=2em]
% pdf交互/链接
\setupinteraction[state=start,color=,]
% pdf书签
% \setupinteractionscreen[option={bookmark}]
\setupinteractionscreen[
    option={doublesided,bookmark},
    width=max,height=max,% necessary for Trim/BleedBox
]
\placebookmarks[DANYUAN,KEWEN,lanmuN,chapter,section,subsection] % 可加入自定义\definelist[mylist]
% 设置引用
% \setupreferencing[left=,right=] % 覆盖左右两侧的引号
% \define[1]\see{\at{}{页}[#1]\about[#1]}% 另见

% %%%%%%%%%%%%%%%%%%%%%%%%%%%%%%%%%%%%%%%%%%%%%%
% 空白、间距和缩进
% %%%%%%%%%%%%%%%%%%%%%%%%%%%%%%%%%%%%%%%%%%%%%%
% 段落间距
\setupwhitespace[none]
% 条目间距(覆盖预设)
% \setupitemgroup[packed]
% 行距设置 A5a类19*20,b类22*22,c类25*25,d类29*28
\setupinterlinespace[line=\getvariable{settings}{interlinespace}]
% 缩进设置
\setupindenting[yes, 2em, first]
% 窄行、缩进设置(每一级的缩进量)
\setupnarrower[left=01em]%
\setupcolumns[n=2, separator=rule]

% %%%%%%%%%%%%%%%%%%%%%%%%%%%%%%%%%%%%%%%%%%%%%%
% 其他设置
% %%%%%%%%%%%%%%%%%%%%%%%%%%%%%%%%%%%%%%%%%%%%%%
% 下画线颜色
\setupthinrules[color=secondColor]
% 表格
\definextable[mytable][offset=0.2em, option={stretch, width}, align={center,lohi}]

% %%%%%%%%%%%%%%%%%%%%%%%%%%%%%%%%%%%%%%%%%%%%%%
% 使用竖排、标点压缩、加注三模块(保持模块顺序)
% %%%%%%%%%%%%%%%%%%%%%%%%%%%%%%%%%%%%%%%%%%%%%%
% 竖排
% \usemodule[vtypeset]
% 
% 标点压缩与支持
\usemodule[zhpunc][pattern=quanjiao, spacequad=0.5, hangjian=false]
% 
% 四种标点压缩方案:全角、开明、半角、原样:
%   pattern: quanjiao(default), kaiming, banjiao, yuanyang
% 行间标点(转换`、,。.:!;?`到行间,pattern建议用banjiao):
%   hangjian: false(default), true
% 加空宽度(角):
%   spacequad: 0.5(default)
% 
% 行间书名号和专名号(\bar实例):
%   \zhuanmh{专名}
%   \shumh{书名}
% 
% 夹注
\usemodule[jiazhu][fontname=tf, fontsize=10.5pt, interlinespace=0.2em]
% default: fontname=tf, fontsize=10.5pt, interlinespace=0.08em(行间标点时约0.2em)
% fontname和fontsize与\switchtobodyfont的对应参数一致
% 夹注命令:
%   \jiazh{夹注}
% %%%%%%%%%%%%%%%%%%%%%%%%%%%%%%%%%%%%%%%%%%%%%%

% %%%%%%%%%%%%%%%%%%%%%%%%%%%%%%%%%%%%%%%%%%%%%%
% 唯一名称(用于自动生成不重复的宏、引用等)
% %%%%%%%%%%%%%%%%%%%%%%%%%%%%%%%%%%%%%%%%%%%%%%
\definecounter[uniquecounter][way=bytext]
% \setnumber[uniquecounter][1]
\def\incrementuniquename{\incrementcounter[uniquecounter]}
\def\uniquename{uniquename\Characters{\rawcountervalue[uniquecounter]}}%

% %%%%%%%%%%%%%%%%%%%%%%%%%%%%%%%%%%%%%%%%%%%%%%
% 存取答案数据
% %%%%%%%%%%%%%%%%%%%%%%%%%%%%%%%%%%%%%%%%%%%%%%
\definedataset[answertable][delay=yes] % 最后实现,会插入页码信息???
\def\saveAnswer[#1,#2]#3{%
    \doifnotmode{answersheet}{%
        % \incrementuniquename%
        % \expandafter\def\csname \strippedcsname\uniquename\endcsname{\someheadnumber[subsubsection][current]}%
        \setdataset[answertable][cmd={#1}, ref={#2}, answer={#3}]%花括号可以避免数据中的符号引起错误
        }%
}

% %%%%%%%%%%%%%%%%%%%%%%%%%%%%%%%%%%%%%%%%%%%%%%
% 处理答案的宏
% %%%%%%%%%%%%%%%%%%%%%%%%%%%%%%%%%%%%%%%%%%%%%%
% 条件文本:校对标记红P、校对标记透明T、答案不着色A
\def\updatePTAColor{%
    \doifmodeelse{answersheet}%
    {\definecolor[PTAColor][aboveColor]}% default 默认颜色
    {\doifmodeelse{proofread}%
        {\definecolor[PTAColor][proofColor]}%校对色
        {\definecolor[PTAColor][transparentColor]}%常态透明
    }%
}
\def\PTA#1{%
    \updatePTAColor%
    \color[PTAColor]{#1}%
}

% 原样输出
\def\answerAsItIs#1{\hskip0pt #1 \hskip0pt} % 有空格,否则插入的quad无效 TODO

% 排印缓存并保存为答案
\def\startBufferAsAnswer{%
    \incrementuniquename%
    \startbuffer[\uniquename]}
\def\bufferAsAnswer#1{\par\page[no]#1}%禁止分页无效 TODO
\def\stopAsAnswer{%
    \getbuffer[\uniquename]%
    \saveAnswer[bufferAsAnswer,]{\getbuffer[\uniquename]}}
% 因为\startbuffer开始后直至\stopbuffer都被认为是buffer,所以\stopbuffer无法包装
% 结束命令为:\stopbuffer\stopAsAnswer
% TDO 也许有控制方法,如此则可以definestartstop,或自定起讫命令

% 行间式增删改标记,以及作答式
\def\Insert[#1,#2]#3[#4]{%
    \dontleavehmode%
    \offset[width=0em,y=-0.7em,location={middle,top}]{\itxx #4 {∨}}%
    % 或用\raise提升,用\hss ∨ \hss居中
    \offset[width=0em,y=- #2 em,location={right,top}]
        {\framedtext[width=#1 em,frame=off,offset=0em,frameoffset=0em,style={\itx\setupinterlinespace[line=1em]\setupalign[flushleft]}]{#4 {#3}}%
    }%
}
\def\InsertCode{\color[secondColor]}
\def\AnswerInsertCode{\PTA}
\def\insert[#1,#2]#3{\Insert[#1,#2]#3[\InsertCode]}
\def\answerInsert[#1,#2]#3{\Insert[#1,#2]#3[\AnswerInsertCode]}

\definebar[delete][overstrike][color=aboveColor, rulethickness=1pt]
\definebar[answerDelete][delete][color=PTAColor]%TODO 直接改变模式不能更新颜色
\def\change[#1,#2]#3#4{\insert[#1,#2]{#3}\delete{#4}}
\def\answerChange[#1,#2]#3#4{\answerInsert[#1,#2]{#3}\answerDelete{#4}\dontleavehmode}

% 行内式增删改
\definebar[insertInlineBar][underbar][color=secondColor, rulethickness=1pt]
\def\insertIn#1{\insertInlineBar{\it #1}}
\def\deleteIn#1{\delete{#1}}
\def\changeIn#1#2{\insertIn{#1}\delete{#2}}% 行内式
% 夹注点评
\def\appreciateIn#1{\jiazh{\color[secondColor]{#1}}}
% %%%%%%%%%
% 自定义标题
% %%%%%%%%%
% 单元
\definehead[DANYUAN][chapter][command={\externalfigure[dummy][width=1cm,height=1cm]}]
\def\danyuan#1{%
    \DANYUAN{#1}%
    % \externalfigure[dummy][width=1cm,height=1cm]%
    \saveAnswer[section,]{#1}%
}
% 等效于(也许更健壮):
% \definehead[danyuan][chapter]
%     [after={\saveAnswer[section,]{\structuretitle}}]

% 课文标题
\definehead[KEWEN][section]
\def\kewenformat#1{\blank \subsubsubject{#1}}
\def\kewen#1{\KEWEN{#1}%
    \saveAnswer[kewenformat,]{\structuretitle}%
}
% \definehead[kewen][section]
%     [after={\saveAnswer[kewenformat,]{\structuretitle}}]

% 栏目/题目分组
% 不上目录的
\definehead[lanmu][subsubject]
    [before={\smallblank}, after={\smallblank}]
% 上目录的
\definehead[lanmuN][subsection]
    [before={\smallblank}, after={\smallblank}]

% 大题目
\definecounter[datinumber][way=bysection]% 大题编号计数器,reset bysection
\def\datiformat#1[#2]{\par\goto{\secondColor{#1}、}[#2]}% 在答案纸中的格式
\def\dati#1{%
    \smallblank \incrementcounter[datinumber] \incrementuniquename%
    %获取实际的值
    % 获取计数器的引用,必须在编号标题中或其后后才能显示,如 subsubsubject{#1、}
    % \saveAnswer[datiformat]{\ref[number][\uniquename]}%
    % 转换计数器的数值为中文数码,以强制替换大题编号
    % 禁止断行要放在最后,TODO 但某些地方仍无效,需要手动再加
    \subsubsection[title={#1}, reference=\uniquename, ownnumber={\convertnumber{cn}{\rawcountervalue[datinumber]}}]%
    \saveAnswer[datiformat,\uniquename]{\structurenumber} \smallblanksamepage%
}

% 小题(一级列表项),用法:\startxiaoti \stopxiaoti
\def\xiaotiformat#1[#2]{\goto{\secondColor{#1}.\nobreak}[#2]}
% 使用before={},比直接将\incrementuniquename插在行头更稳健(会受到周围空格的影响)
\definestartstop[xiaoti][
   before={%
        \incrementuniquename%
        \startitem[\uniquename]%
        \saveAnswer[xiaotiformat,\uniquename]{\ref[number][\uniquename]}},
   after={\stopitem}]
% \def\xiaoti{\item \saveAnswer[xiaotiformat]{\currentstructurelistnumber}} %数字错误

% 小小题,二级列表项,用法:\startxxiaoti \stopxxiaoti
\def\xiaoxiaotiformat#1[#2]{\goto{\secondColor{(#1)\nobreak}}[#2]}
\definestartstop[xxiaoti][
   before={%
        \incrementuniquename%
        \startitem[\uniquename]%
        \saveAnswer[xiaoxiaotiformat,\uniquename]{\ref[number][\uniquename]}},
   after={\stopitem}]

% %%%%%%%%%
% 题型
% %%%%%%%%%
% 下画线完形填空answerUnderbar
% !!!标点符号处理过程导致下画线断开,或禁则失效 TODO 用\blackrule和\hspace 模拟
% 可能是标点模块删除标点前后的空白造成的
% 临时处理方法是只在填空前面加kern
\definebar[clozeBar][underbar][continue=yes, rulethickness=0.25pt, color=secondColor, offset=-0.5]
\def\answerU#1{%
    \startbar[clozeBar]%
    \color[transparentColor]{#1}\PTA{#1}% 
    % 下面的方法会导致断句困难
    % \scratchdimen\widthofstring{#1}\relax % 获取文本宽度
    % \zwj\kern1\scratchdimen \color[black]{#1}% \kern.5\scratchdimen\zwj % U+200D ZERO WIDTH JOINER
    \stopbar%
    \saveAnswer[answerAsItIs,]{#1}%
}%

% 就地填空(没有下划线)
\def\answerFill#1{%
    \color[transparentColor]{#1}\PTA{#1}% 
    \saveAnswer[answerAsItIs,]{#1}%
}%

% 回答画掉slashout
\def\answerS#1{%
    \setbox0=\hbox{#1}%
    % 阻止切换到v模式以免断行
    \dontleavehmode\copy0\kern-\wd0\hbox to \wd0{\hfil\PTA{\backslash}\hfil}%
    \saveAnswer[answerS,]{#1}%
}

% 回答勾选check
\def\check#1{%
    \setbox0=\hbox{#1}%
    % 阻止切换到v模式以免断行
    \dontleavehmode\copy0\kern-\wd0\hbox to \wd0{\hfil\PTA{\TrueMark}\hfil}%
}
\def\answerC#1{%
    \check{#1}%
    \saveAnswer[answerC,]{#1}%
}

% 回答画选checkUnderbar
\definebar[checkBar][clozeBar][color=PTAColor]
\def\checkU#1{%
    \startbar[checkBar]%
    #1% 
    \stopbar%
}
\def\answerCU#1{%
    \updatePTAColor%
    \checkU{#1}%
    \saveAnswer[answerCUformat,]{#1}%
}
\def\answerCUformat#1{\hskip0pt \answerCU{#1} \hskip0pt}

% 回答序号
\def\answerN#1{%
    \incrementuniquename%
    \startitem[\uniquename]#1%
    \raise 1em\vbox to 0em{\hbox to 0em{\itx\PTA{\ref[number][\uniquename]}}}%
    \stopitem%
    \saveAnswer[answerAsItIs,]{\ref[number][\uniquename]}%
}

% 回答“略” omit
\def\LableO{(略)}
\def\answerLO{%
    \dontleavehmode%
    \raise 1em\vbox to 0em{\hbox to 0em{\itx\PTA{\LableO}}}%
    \saveAnswer[answerAsItIs,]{\LableO}%
}

% 回答“示例:”example
\def\LableE{示例:}
\def\answerLE{%
    \dontleavehmode%
    \raise 1em\vbox to 0em{\hbox to 0em{\itx\PTA{\LableE}}}%
    \saveAnswer[answerAsItIs,]{\LableE}%
}

% 回答“文中练习”inline TODO 合并多题的提示
\def\LableI{〖文中练习〗}
\def\answerLI{%
    \dontleavehmode%
    \raise 1em\vbox to 0em{\hbox to 0em{\itx\PTA{\LableI}}}%
    \saveAnswer[answerAsItIs,]{\LableI}%
}

% 回答“见文中练习”inline as given above
\def\LableIA{(见\LableI)}
\def\answerLIA{%
    \dontleavehmode%
    \raise 1em\vbox to 0em{\hbox to 0em{\itx\PTA{\LableIA}}}%
    \saveAnswer[answerAsItIs,]{\LableIA}%
}

% 说明文字 TODO 宽度错误
\def\note#1{%
    \dontleavehmode%
    \inoutermargin[style={\itx\setupinterlinespace[line=1em]\setupalign[flushleft]}]{\PTA{#1}}%
}

% 括号
% TODO 测量宽度;允许输入宽度
\def\brackets#1{(\nobreak\enskip \PTA{#1} \enskip)}
% 括号(行末)
\def\rBrackets#1{\hfill \brackets{#1}}

% 括号中画勾
\def\answerT{%
    \brackets{\TrueMark}%
    \saveAnswer[answerAsItIs,]{\TrueMark}%
}
% 括号(行末)中画勾
\def\answerTR{%
    \rBrackets{\TrueMark}%
    \saveAnswer[answerAsItIs,]{\TrueMark}%
}

% 括号中画叉
\def\answerF{%
    \brackets{\FalseMark}%
    \saveAnswer[answerAsItIs,]{\FalseMark}%
}
% 括号(行末)中画叉
\def\answerFR{%
    \rBrackets{\FalseMark}%
    \saveAnswer[answerAsItIs,]{\FalseMark}%
}

% 填入括号
\def\answerB#1{%
    \brackets{#1}%
    \saveAnswer[answerAsItIs,]{#1}%
}
% 填入括号(行末)
\def\answerBR#1{%
    \rBrackets{#1}%
    \saveAnswer[answerAsItIs,]{#1}%
}

% 看拼音,在田字格中写字词 TODO 字数校验
% 导入lua模块MPwordslink.lua
\startluacode
    require "envs/Pinyinge"
\stopluacode
\def\pinyinge[#1][#2]{%
    \dontleavehmode
    \bbox{%底线对齐(默认为基线对齐)
    \startMPcode
        lua.MP.tianzige("#1","#2") ;
    \stopMPcode
    }%
    \saveAnswer[answerAsItIs,]{#2}%
}

% %%%%%%%%%%%%%%%%%%%%%%%%%%%%%%%%%%%%%%%%%%%%%%
% 排式
% %%%%%%%%%%%%%%%%%%%%%%%%%%%%%%%%%%%%%%%%%%%%%%
% 边注
\defineframedtext[bianzhu]
    [corner=round,
    width=12.5em,
    indenting={yes, 2em, first}, % 无效
    offset=0.25em, % 内容向内偏置(可用loffset等分别指定)
    frameoffset=0em, % 外框向外偏置
    location=hanging, % depth, low, hanging, none
    style={\it}]
\definefloat[bianzhumezzo][bianzhumezzos][intermezzo]
\setupcaption[bianzhumezzo][location=none] % 不加标题
\define[1]\bz{%
    \placefloat[bianzhumezzo][right,nonumber]{不显示标题}{\bianzhu{#1}}}

% 编号段落
\newcounter\Paracount
\def\Paragraphnumber%
    {\increment\Paracount 
    \Paracount.~}
\def\startParagraphNumbers%
    {\par \begingroup \appendtoks \Paragraphnumber \to \everypar}
\def\stopParagraphNumbers%
    {\par \endgroup}

% 带括号的多行组
% TODO 调整前头的空间;自动计算高度;绘制大括号
% vbox式只能用分栏的形式并排,以获取vbox的宽度
% 换成图形则可用combination并排
\def\kuohaozu[#1,#2]#3{%右侧部分的宽度(em)、括号缩放倍数、内容
    % 用lua逐行处理,未完成
    % \startluacode
    %     MP = MP or {}
    %     % MP.lines = string.splitlines([=[#1]=]) --无效
    %     % MP.lines = string.explode([=[#1]=], [=[\\]=]) --乱
    %     MP.lines = string.split([=[#1]=], [=[\\]=])
    % \stopluacode
    % 
    % \startMPcode
    % draw textext.lft("\[") yscaled 2;
    % for i=1 upto lua("mp.size(MP.lines)"):
    %     draw textext.rt(lua("mp.quoted(MP.lines[" & decimal i & "])"));
    % endfor ;
    % \stopMPcode
    % 
    \setbox0=\hbox{\framedtext[%
        offset=default,%
        location=depth,%
        frame=off,%
        width=#1em,%
        % autowidth=force,% yes no force 无效
        ]{#3}}
    \dontleavehmode%
    \startMPcode
    % 用加框文本可以实现更复杂的控制
    draw textext.lft("\[") yscaled #2;
    draw textext.rt("\copy0"); % 不能将\framedtext代码直接放在这里,否则会归集两次答案,原因不明
    % 
    % 简易方式
    % draw textext.rt("\hbox to #1em{\vbox{#3}}"); -- BUG 答案会归集两次
    % draw textext.lft("\strut\[") yscaled #2;
    \stopMPcode
}

% 连线题
\startluacode
    require "envs/Lianxian"
\stopluacode
% #1三列宽度[8em,2cm,1em],#2后项顺序[231],#3前项,后项#4
\def\lianxian[#1][#2]#3#4{%
    \dontleavehmode
    \startMPcode
        lua.MP.lianxian("#1","#2","#3","#4") ;
    \stopMPcode
    % \saveAnswer[answerAsItIs,]{#2}%
}

% 思维导图
\startluacode
    require "envs/Mindmap"
\stopluacode
% #1三列宽度[8em,2cm,1em],#2后项顺序[231],#3前项,后项#4
\def\lianxian[#1][#2]#3#4{%
    \dontleavehmode
    \startMPcode
        lua.MP.mindmap("#1","#2","#3","#4") ;
    \stopMPcode
    % \saveAnswer[answerAsItIs,]{#2}%
}

% 点拨文本框
% 随机曲线风格
\startuseMPgraphic{FunnyFrame}
    picture p ; numeric o ; path a, b ; pair c ;
    p := textext.rt(\MPstring{FunnyFrame}) ;
    a := unitsquare xyscaled(OverlayWidth,OverlayHeight) ;
    o := BodyFontSize ;
    p := p shifted (2o,OverlayHeight-ypart center p) ;
    drawoptions (withpen pencircle scaled 1pt withcolor \MPcolor{secondColor}) ;
    b := a randomized (o/2) ;
    fill b withcolor white ; draw b ;
    b := (boundingbox p) randomized (o/8) ;
    fill b withcolor white ; draw b ;
    draw p withcolor black;
    setbounds currentpicture to a ;
\stopuseMPgraphic

% 圆角方形
% \startuseMPgraphic{FunnyFrame}
%     picture p ; numeric o ; path a, b ; pair c ;
%     p := textext.rt(\MPstring{FunnyFrame}) ;
%     o := BodyFontSize ;
%     a := unitsquare xyscaled(OverlayWidth,OverlayHeight) ;
%     p := p shifted (2o,OverlayHeight-ypart center p) ;
%     pickup pencircle scaled OverlayLineWidth ;
%     b := a superellipsed .95 ;
%     fill b withcolor \MPcolor{lightSecondColor} ;
%     draw b withcolor \MPcolor{aboveColor} ;
%     b := (boundingbox p) superellipsed .95 ;
%     fill b withcolor \MPcolor{lightSecondColor} ;
%     draw b withcolor \MPcolor{aboveColor} ;
%     draw p withcolor black ;
%     setbounds currentpicture to a ;
% \stopuseMPgraphic

\defineoverlay[FunnyFrame][\useMPgraphic{FunnyFrame}]
\defineframedtext[FunnyText][frame=off,background=FunnyFrame]
\setupframedtexts
    [FunnyText]
    [backgroundcolor=lightgray,
    % foregroundcolor=white,
    framecolor=darkred,
    rulethickness=2pt,
    offset=\bodyfontsize,
    before={\blank[big,medium]},
    after={\blank[big]},
    width=\textwidth]
\def\StartDianbo{\startFunnyText}
\def\StopDianbo {\stopFunnyText }

\def\DianboTitle#1%
{\setMPtext{FunnyFrame}{\hbox spread 1em{\hss\strut#1\hss}}}
\setMPtext{FunnyFrame}{} % initialize the text variable

% 
\def\bgframedtext[#1]#2{%
\startuseMPgraphic{frame}
    path p; p := unitsquare xyscaled(OverlayWidth,OverlayHeight);
    draw p randomized 2.5mm
    withcolor transparent (1, 0.4 randomized 0.25, \MPcolor{secondColor})
    withpen pencircle scaled 2pt;
\stopuseMPgraphic
\defineoverlay[frame][\useMPgraphic{frame}] % 可以多层: \startoverlay ... \stopoverlay
\dontleavehmode\framedtext[frame=off,background={frame},width=#1em]{#2}%
}

% %%%%%%%%%%%%%%%%%%%%%%%%%%%%%%%%%%%%%%%%%%%%%%
% 资源
% %%%%%%%%%%%%%%%%%%%%%%%%%%%%%%%%%%%%%%%%%%%%%%
\useMPlibrary[dum]%占位图库
% \setupexternalfigures[location={local,default}]
% \useexternalfigure[sine][sin.pdf][width=3cm]
% \externalfigure[sine]

% %%%%%%%%%%%%%%%%%%%%%%%%%%%%%%%%%%%%%%%%%%%%%%
% 默认开启的信息
% %%%%%%%%%%%%%%%%%%%%%%%%%%%%%%%%%%%%%%%%%%%%%%
\replacemissingcharacters % 缺字体时替换为方框标记

\stopenvironment