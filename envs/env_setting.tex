%%%% 全局配置
\startenvironment

% 校对模式
\definemode[proofread][yes]
\enablemode[proofread]
% \doifmodeelse{proofread}
%     {}
%     {}

%%%% 纸张和版面
\definepapersize[sheet][width=145mm, height=210mm]
% Create two A4-derived paper sizes
\definepapersize[standing][A4][sheet]
\setuppapersize[standing]
\setuplayout[
    marking=on, % 裁切线,设置较大幅面打印纸才能显示
    location=middle,
]

% 中文配置
\mainlanguage[cn]
\language[cn]
\setscript[hanzi] % 汉字处理脚本(断行)
\setupalign[hanging,hz] %行末标点悬挂

% 汉字字体配置
\usetypescriptfile[mscore]
\usebodyfont   [mschinese,14pt]
% \usebodyfont   [mschinese-light,12pt]
% \setupbodyfont [mschinese-literate,12pt]
% \usebodyfont   [mschinese-literate,12pt]
% \definebodyfontenvironment[24pt]
% \definebodyfontenvironment[18pt]
% 定义字体
% \definefont[kaiti][name:kaiti*default at 24pt]
\definefont[checkMarkFont][name:dejavusansmono*default]
\define\checkMark{{\checkMarkFont ✔}}
\define\crossMark{{\checkMarkFont ✗}}

% 结构转换集合(序号集合)
\definestructureconversionset[cnconversion]
    [R,   % part: Rome
     cn,  % chapter:中文数字
     ,   % section: none defined (use fallback)
     ,   % subsection: 
     cn]   % subsubsection: 中文数字
    [r] % Default setting

% 重置/不重置集合
\defineresetset[myresetset]
    [1,1,1,1,0,1,1] % 对应7级标题
    [1] % 1=reset, 0=don't reset
\setupheads
    [part,chapter,section,subsection,subsubsection]
    [%
    % before=, % 覆盖上下行插入的空格
    after=, % 覆盖上下行插入的空格
    sectionconversionset=cnconversion,
    number=no, % 序号
    indentnext=yes, % 其后首行缩进
    sectionsegments=current, % 序号段:只包括当前序号
    % sectionresetset=myresetset, %
    ]
\setupheads
    [subsection,subsubsection]
    [%
    before=, % 覆盖上下行插入的空格
    ]


% 配置标题、目录格式
% 单元,新页起 , page=yes新页/left偶数页/right奇数页(默认)
\setuphead[chapter][page=yes,style={\setupalign[middle]\tfb}]
%  课目
\setuphead[section][style={\setupalign[middle]\tfa}]
% 栏目
\setuphead[subsubsubject][style={\bf}]
% 题目
\setuplabeltext[cn][subsubsection={{},{、}}] % 前后标签
\setuphead[subsubsection][style={\tf}, color=middlecyan, number=yes,distance=0pt]

% 自订文章标题(6级)、作者(7级)
\definehead[centersection][subsubsubsubject][align=middle, before=, after=,]
\definehead[authorsection][subsubsubsubsubject][align=middle, before=, after=,]
% \definehead[leftsection][subsubsubject][align=flushleft]
\def\essayhead[#1][#2]{%
    \centersection{#1}%
    \doifnotempty{#2}{\authorsection{#2}}%
    }
% \define[2]\rhymehead{\leftsection{#1}\leftaligned{\itx #2}\blank[0cm]}
% \define[2]\classicalpoemhead{\centersection{#1}\middlealigned{\itx #2}\blank[0cm]}

% 短文标题
\def\centralTitle[#1][#2]{%
    \subsubsubject{\tf }%
}

\setupcolors[state=start]
\setupheadtext[content=目录]
\setupcombinedlist[content][list={chapter,section},alternative=c]%
\setuplist[section][width=10mm, style=bold]
\setuplist[subsection][width=20mm]
\setuplist[subsubsection][width=20mm, style=slanted, pagestyle=normal]
% pdf交互/链接
\setupinteraction[state=start]

% 设置页码格式
\setuppagenumbering[
    alternative=doublesided,
    location={header, margin},
]

% 段落间距
\setupwhitespace[none]
% 条目间距(覆盖预设)
% \setupitemgroup[packed]
% 行距设置
\setupinterlinespace[line=1.5em]
% 缩进设置
\setupindenting[yes, 2em, first]
% 窄行、缩进设置(每一级的缩进量)
\setupnarrower[left=01em]%
\setupcolumns[n=2, separator=rule]


% % 引用、交互
% \setupreferencing[left=,right=] % 覆盖左右两侧的引号
% \setupinteraction   [state=start, color=,contrastcolor=black,style=]
% \define[1]\see{\at{}{页}[#1]\about[#1]}% 另见

% % 页眉
% % \setupheadertexts[{\getmarking[subsubject]}][][][{\getmarking[subsubject]}]
% \setupheadertexts%
%     []%
%     [{\ReadFile{header.lua}}]%
%     [{\ReadFile{header.lua}}]%
%     []%


% %%%% 存取数据

% \definedataset[zicitable][delay=yes] % 最后实现,会插入页码信息

% %%%% 宏

% % \define[3]\zi{%
% %     \startnarrower[left]%
% %         \setupindenting[yes,-1em]%
% %         \setupinitial[n=2,text=#1]%
% %         \placeinitial #2 \par%
% %         \forgetinitial%
% %         \setupindenting[yes,2em] #3%
% %     \stopnarrower%
% % }

% % 单字词条 \zi{单字词条}{音}{正文}{另见}
% % \defineexpandable
% \define[4]\zi{%
%     \startnarrower[left]%
%         \newcount\myhanglinenum \myhanglinenum=2% 字头绕排行数
%         \newdimen\mytitleindent \mytitleindent=-1em% 字头缩进量
%         \newdimen\hangingdist \hangingdist=-0.5em% 绕排间距

%         \setbox0=\hbox{\tfd \bf #1}% 字头
%         \newdimen\titleright \titleright=\wd0% 字头宽度
%         \advance\titleright by \mytitleindent%
%         \advance\titleright by -\hangingdist%
        
%         \setupindenting[yes,\mytitleindent]% 缩进
%         % 字头绕排
%         \starthanging[location=left,n=\myhanglinenum,distance=\hangingdist]{%
%             \reference[#1-#2]{#2}% 引用锚点
%             \setdataset[zicitable][title=#1, pinyin=#2, class=zi]% 登记在字词表中,不用键名
%             \box0% 字头
%         }%
%             {\hskip -\mytitleindent #2 #3}\par% 字头解释,字头缩进后需要用hskip和hangingdist修正
%             \advance \myhanglinenum by -\prevgraf% 减去上一个段落的行数
%             % 如果实际绕排1行,还有1个空行,则后行跟上、加缩进
%             \ifnum \myhanglinenum=1%
%                 \doifsomethingelse{#4}
%                     {\setupindenting[yes,\titleright]另见#4}%
%                     {\dorecurse{\myhanglinenum}{\blank}}%
%             \else%
%                 \doifsomethingelse{#4}
%                     {\setupindenting[yes,2em]另见#4}%
%             \fi%
%         \stophanging%
%     \stopnarrower%
% }

% % 多字词条 \zi{多字词条}{音}{正文}{另见}
% \define[4]\ci{%
%     \startnarrower[left]%
%         \setupindenting[yes,-1em]%
%         \reference[#1-#2]{#2}% 引用锚点
%         \setdataset[zicitable][title=#1, pinyin=#2, class=ci]% 登记在字词表中,不用键名
%         \hbox{\bf #1} #2 #3 \par% 词头和词头解释
%         \doifsomethingelse{#4}
%         {\setupindenting[yes,2em]另见#4}% 另见
%         \forgetinitial%
%     \stopnarrower%
% }

%%%%%%%%%%%%% 使用模块(保持顺序) %%%%%%%%%%%%%
% 竖排
% \usemodule[vtypeset]
% 
% 标点压缩与支持
\usemodule[zhpunc][pattern=quanjiao, spacequad=0.5, hangjian=false]
% 
% 四种标点压缩方案:全角、开明、半角、原样:
%   pattern: quanjiao(default), kaiming, banjiao, yuanyang
% 行间标点(转换`、,。.:!;?`到行间,pattern建议用banjiao):
%   hangjian: false(default), true
% 加空宽度(角):
%   spacequad: 0.5(default)
% 
% 行间书名号和专名号(\bar实例):
%   \zhuanmh{专名}
%   \shumh{书名}
% 
% 夹注
% \usemodule[jiazhu][fontname=tf, fontsize=10.5pt, interlinespace=0.2em]
% default: fontname=tf, fontsize=10.5pt, interlinespace=0.08em(行间标点时约0.2em)
% fontname和fontsize与\switchtobodyfont的对应参数一致
% 夹注命令:
%   \jiazh{夹注}
%%%%%%%%%%%%%%%%%%%%%%%%%%%%%%%%%%%%%%%%%%%%%%%%%%%%

% 着重号
\definebar[zhuozhong][text=\lower0.6em\hbox{~·}, repeat=no]

% 斜线删除slashout
\def\slashout#1{%
    \setbox0=\hbox{#1}%
    \hbox{\copy0 \kern-0.625\wd0 \hbox to 0.5\wd0{\backslash}}
}

% 完形填空
% !!!标点符号处理过程导致下划线丢失,或禁则失效
% 可能是标点模块删除标点前后的空白造成的
% 临时处理方法是只在填空前面加kern
\definebar[clozeBar][underbar][continue=yes, color=black, offset=-0.5]
\def\cloze #1{%
    \startbar[clozeBar]%
    #1☆#1%
    % 下面的方法会导致断句困难
    % \scratchdimen\widthofstring{#1}\relax % 获取文本宽度
    % \zwj\kern1\scratchdimen \color[black]{#1}% \kern.5\scratchdimen\zwj % U+200D ZERO WIDTH JOINER
    \stopbar%
}%

% 边注
\defineframedtext[bianzhu]
    [corner=round,
    width=12.5em,
    indenting={yes, 2em, first}, % 无效
    offset=0.25em, % 内容向内偏置(可用loffset等分别指定)
    frameoffset=0em, % 外框向外偏置
    location=hanging, % depth, low, hanging
    style={\it}]
\definefloat[bianzhumezzo][bianzhumezzos][intermezzo]
\setupcaption[bianzhumezzo][location=none] % 不加标题
\define[1]\bz{%
    \placefloat[bianzhumezzo][right,nonumber]{不显示标题}{\bianzhu{#1}}}

% 编号段落
\newcounter\Paracount
\def\Paragraphnumber%
    {\increment\Paracount 
    \Paracount.~}
\def\startParagraphNumbers%
    {\par \begingroup \appendtoks \Paragraphnumber \to \everypar}
\def\stopParagraphNumbers%
    {\par \endgroup}


% 田字格
% 导入lua模块MPwordslink.lua
\startluacode
    require "envs/Pinyinge"
\stopluacode
\def\pinyinge[#1][#2]{
    \startMPcode
        lua.MP.tianzige("#1","#2") ;
    \stopMPcode
}

% 括号分组
% TODO 调整前头的空间;自动计算高度;绘制大括号
% vbox式只能用分栏的形式并排,以获取vbox的宽度
% 换成图形则可用combination并排
\def\kuohaozu#1{%
    % 用lua逐行处理,未完成
    % \startluacode
    %     MP = MP or {}
    %     % MP.lines = string.splitlines([=[#1]=]) --无效
    %     % MP.lines = string.explode([=[#1]=], [=[\\]=]) --乱
    %     MP.lines = string.split([=[#1]=], [=[\\]=])
    % \stopluacode
    % \startMPcode
    % draw textext.lft("\[") yscaled 2;
    % for i=1 upto lua("mp.size(MP.lines)"):
    %     draw textext.rt(lua("mp.quoted(MP.lines[" & decimal i & "])"));
    % endfor ;
    % \stopMPcode

    \startMPcode
    % 用加框文本可以实现更复杂的控制
    % draw textext.lft("\[") yscaled 2;
    % draw textext.rt("\framedtext[%
    %     offset=default,%
    %     location=depth,%
    %     frame=off,%
    %     % width=7em,% local fit max broad fixed dimension
    %     % autowidth=force,% yes no force 无效
    %     ]{#1}");

    % 简易方式
    draw textext.lft("\strut \[") yscaled 2;
    draw textext.rt("\vbox{#1}")
    \stopMPcode
}

% 表格
\definextable[mytable][offset=0.2em, option={stretch, width}, align={center,lohi}]

\replacemissingcharacters % 缺字替换

\stopenvironment